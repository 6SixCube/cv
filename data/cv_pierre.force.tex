\documentclass[11pt,a4paper,sans]{moderncv}
\moderncvstyle{casual} 
\moderncvcolor{blue}
\setlength{\hintscolumnwidth}{2cm} 
\usepackage[utf8]{inputenc}
\usepackage[scale=0.8]{geometry}
\usepackage{helvet}
\usepackage[french]{babel}
\name{Pierre}{FORCE}
\title{Ingénieur Qualité} 
\address{55, place des mimosas}{69\,290 Craponne}{France}
\phone[mobile]{06~09~05~69~71}
\email{force.pierre@gmail.com}
\photo[64pt][0.4pt]{data/pierreforce.jpg}
%\quote{}
\begin{document}
\makecvtitle
\section{Formation}
\cventry{2003--2006}{Diplome Universitaire de technologie}{IUT université Clermont Auvergne}{Clermont-Ferrand (63)}{}{Spécialité Réseaux et Télécom}
\cventry{2000--2003}{baccalauréat Série Scientifique}{Lycée Godefroy de bouillon}{Clermont-Ferrand (63)}{}{Option Science de l'ingénieur}
\section{Experience professionnelle}
\cventry{2016--2019}{Ingénieur Qualité}{Stormshield}{Lyon (69)}{Développement d'un logiciel d'administration centralisé de pare-feu : Stormshield Managment Center [SMC]}%
{%
\begin{itemize}%
\item Validation fonctionnelle du produit SMC :
  \begin{itemize}%
  \item Validation des nouvelles fonctionnalitées du Produit.
  \item Reflexion et mise en place de maquettes complexes permettant de prouver le bon fonctionnement du produit dans le cadre de son évolution. 
  \item Reflexion et mise en place de de tests automatiques permattant de prouver le bon fonctionnement du produit dans le cadre de son évolution. 
  \item Reflexion et écriture de plan tests manuels permattant de prouver le bon fonctionnement du produit ainsi que de transmettre son fonctionnement.
  \end{itemize}
\end{itemize}}
\begin{itemize}%
\item Évolution des Process de validation (Méthode Agile : Scrum):
\end{itemize}
\cventry{2010--2013}{Ingénieur d'études de projets}{Cerdux}{Reims}{}{%
\begin{itemize}
\item Études de développement d'installations ou de systèmes
industriels automatisés pour définir la solution optimale dans le
contrôle des mouvements des machines ;
\item Rédaction et suivi d'offres proposant des solutions techniques
selon les besoins ;
\end{itemize}}
\section{Langues}
\cvitemwithcomment{Anglais}{Lu, parlé, écrit}{un commentaire si besoin}
\cvitemwithcomment{Allemand}{Scolaire}{Idem}
\section{Compétences informatiques}
\cvdoubleitem{Java}{blabla, blabla}{C++}{blabla, blabla}
\cvdoubleitem{Php}{blabla, blabla}{Pascal}{blabla, blabla}
\cvdoubleitem{\LaTeX}{blabla, blabla}{Python}{blabla, blabla}
\section{Centres d'intérêts}
\cvitem{hobby 1}{Description}
\cvitem{hobby 2}{Description}
\end{document}
