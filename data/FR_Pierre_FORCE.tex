\documentclass[10pt,a4paper,sans]{moderncv}
\moderncvstyle{casual} 
\moderncvcolor{blue}
\setlength{\hintscolumnwidth}{2cm}
\usepackage[utf8]{inputenc}
\usepackage[scale=0.9, top=0.3cm, bottom=2.5cm]{geometry}
\usepackage{helvet}
\usepackage[french]{babel}
\usepackage [T1] {fontenc}
\frenchsetup{ItemLabels=\textendash}
\frenchsetup{ItemLabeli=\textbullet}
\usepackage{xspace}
\usepackage{tabto}
\newcommand{\g}[1]{\og #1 \fg}
\definecolor{myblue}{rgb}{0.22,0.45,0.70}

\name{Pierre}{FORCE}
\title{Ingénieur Systèmes et Réseaux} 
\address{57, rue de la forêt}{L1534}{Luxembourg}
\phone[mobile]{+33 6 09 05 69 71}
\email{force.pierre@gmail.com}
\photo[64pt][0.4pt]{data/pierreforce.jpg}
\homepage{https://6sixcube.github.io/cv/FR_Pierre_FORCE.html}
\extrainfo{36 ans - Permis B}

\begin{document}
\makecvtitle
\section{Experience professionnelle}
	\cventry{2016--2019}{Ingénieur Qualité}{Stormshield, Lyon (France)}{Participation au développement d'un logiciel d'administration centralisée de firewall : \g{Stormshield Management Center}(SMC)}{}
		{
			\begin{itemize}
				\item \textbf{Validation fonctionnelle du produit SMC :}
			\end{itemize}
			Mise en place de maquettes de réseaux complexes, écriture de plans de tests et scénarios fonctionnels et techniques, réalisation de tests automatisés dans le but d'attester du bon fonctionnement du logiciel dans son évolution.\\
			\textcolor{myblue}{Compétences : Réseaux, Système, Firewall, Virtualisation, Scripting.}\\
            \textcolor{myblue}{Technologies : Stormshield, Linux, Python, Bash, Robotframwork, Docker, VmWare, Qemu-kvm (Ganéti), Git, Git flow, GitLab, Jenkins, HTML, API.}
			\begin{itemize}
				\item \textbf{Organisation du travail dans le cadre de la méthode \g{Agile Scrum}:}
			\end{itemize}
			Gestion d'équipe, animation des réunions, reporting au responsable produit, recrutement, réflexion sur l'amélioration des procédures et leur mise en place.\\
			\textcolor{myblue}{Compétences : Scrum Master, Organisation, Process, Intégration, Gestion d'équipe.}
		}
	\cventry{2008--2016}{Administrateur Systèmes et Réseaux}{Stormshield, Lyon (France)}{Administration, maintenance et supervision du parc informatique et des utilisateurs de Stormshield}{}
	    {
		    \begin{itemize}
				\item \textbf{Gestion de l'infrastructure de Stormshield :}
			\end{itemize}
			Conception et mise en place de solutions permettant d'améliorer l'infrastructure des sites de Stormshield, coordination de projets de mise en place de nouveaux services informatiques, gestion des sauvegardes, maintenance des logiciels et matériels, gestion des incidents utilisateurs et des postes de travail, responsable des partenaires et des achats.\\
			\textcolor{myblue}{Compétences : Réseaux, Firewall, Virtualisation, Système, Stockage SAN et NAS.}\\
			\textcolor{myblue}{Technologies : Stormshield, Spanning Tree, Linux, Windows, Powershell, Bash, VmWare, Glpi, Apache, MySQL, Microsoft Exchange Server, Zimbra, Microsoft Active Directory, GPO, RAID, PKI.}
		} 	
	\cventry{2007--2008}{Prestataire de Service Informatique}{FlowLine, Clermont-Ferrand (France)}{Audit, conseil et mise en place de solutions IT chez les clients}{}
	{
			\textcolor{myblue}{Compétences : Réseaux, Systèmes, Firewall, Virtualisation, Scripting.}\\
			\textcolor{myblue}{Technologies : Arkoon, Linux, Windows, Bash, VmWare, Kerio mail.}
	}
\cventry{2006--2007}{Stages effectués dans le cadre des études}{}{}{}
{
       \begin{itemize}
			\item \textbf{Oberthure Gaming Technologies, Montréal (Canada) : }Intégration au sein du service informatique de l'entreprise. Sujet de stage : Étude et mise en place d'un plan de reprise d'activité.
            \item \textbf{Marketik, Saint-Étienne (France) :} Développement d'une application web de gestion des produits, clients.
       \end{itemize}
}

\section{Formation}
	\cventry{2006}{Diplôme Universitaire de Technologies}{IUT Université Clermont Auvergne (France)}{}{Spécialité Réseaux et Télécommunications}{}
	\cventry{2003}{Baccalauréat Série Scientifique}{Lycée Godefroy de Bouillon, Clermont-Ferrand (France)}{}{Option Sciences de l'Ingénieur}{}
\section{Langues}
\cvdoubleitem{\textbf{Anglais}}{Niveau C1 (CECRL)}{\textbf{Français}}{Langue maternelle}
\section{Compétences informatiques}
\cvdoubleitem{\textbf{Réseaux}}{Switch, Spanning Tree, Routage}{\textbf{Systèmes}}{Linux (Debian, Ubuntu), Windows XP -> 10}
\cvdoubleitem{\textbf{Virtualisation}}{Vmware, VCenter, QEMU-KVM (Ganeti)}{\textbf{Versionning}}{Git, Git flow, Git Lab, Jenkins}
\cvdoubleitem{\textbf{Firewall}}{Arkoon, Netasq, Stormshield}{\textbf{Script}}{Python, Bash, Programation objet}
\cvdoubleitem{\textbf{Chiffrement}}{PKI, X509, OpenSSL}{\textbf{Inter-connexion}}{MPLS, VPN IPSEC,VPN SSL}
\cvdoubleitem{\textbf{Web}}{MySQL, HTML, API, PHP}{\textbf{Infra}}{SAN, NAS, RAID, Dell, Server}
\end{document}