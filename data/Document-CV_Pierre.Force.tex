\documentclass[10pt,a4paper,sans]{moderncv}
\moderncvstyle{casual} 
\moderncvcolor{blue}
\setlength{\hintscolumnwidth}{2cm}
\usepackage[utf8]{inputenc}
\usepackage[scale=0.9, top=0.3cm, bottom=2.5cm]{geometry}
\usepackage{helvet}
\usepackage[french]{babel}
\usepackage [T1] {fontenc}
\frenchsetup{ItemLabels=\textendash}
\frenchsetup{ItemLabeli=\textbullet}
\usepackage{xspace}
\usepackage{tabto}
\newcommand{\g}[1]{\og #1 \fg}
\definecolor{myblue}{rgb}{0.22,0.45,0.70}

\name{Pierre}{FORCE}
\title{Ingénieur Qualité} 
\address{57, place des mimosas}{69\,290 Craponne}{France}
\phone[mobile]{06~09~05~69~71}
\email{force.pierre@gmail.com}
\photo[64pt][0.4pt]{data/pierreforce.jpg}

\begin{document}
\makecvtitle
\section{Formation}
	\cventry{2003--2006}{Diplôme Universitaire de technologie}{IUT université Clermont Auvergne}{Clermont-Ferrand (63)}{}{Spécialité Réseaux et Télécommunication}
	\cventry{2000--2003}{baccalauréat Série Scientifique}{Lycée Godefroy de Bouillon}{Clermont-Ferrand (63)}{}{Option Science de l'ingénieur}
\section{Experience professionnelle}
	\cventry{2016--2019}{Ingénieur Qualité}{Stormshield}{Lyon (69)}{Développement d'un logiciel d'administration centralisé de pare-feu : \g{Stormshield Management Center} (SMC)}
		{
			\begin{itemize}
			\item \textit{Validation fonctionnelle du produit SMC :}\\
			\tabto{0.3cm} Mise en place de maquettes de réseaux complexes, Écriture de plans de tests et scénario fonctionnels et techniques, Écriture de Tests automatisés dans le but de prouver le bon fonctionnement du logiciel dans son évolution.\\
			\textcolor{gray}{\emph{\scriptsize{Compétence : Réseaux, système, pare-feu, Virtualisation, Scripting}}}\\
			\textcolor{gray}{\emph{\scriptsize{Technologie : Stormshield, linux, Python, bash, robotframwork, docker, VmWare, quemu-kvm (ganéti)}}}
			\end{itemize}
			\begin{itemize}
			\item \textit{Méthode de travail dans le cadre de la méthode \g{Agile Scrum}:}\\
			\tabto{0.3cm} Gestion du quotidien de son équipe, animation des réunions, Reporting au responsable produit, recrutement, Reflexion et mise en place sur l'amélioration des procédures.\\
			\textcolor{gray}{\emph{\scriptsize{Compétence : Leader, Organisation, Process}}}
			\end{itemize}
		}
	\cventry{2008--2016}{Administrateur Systèmes et réseaux}{Stormshield}{Lyon (69)}{Administration, maintenance et supervision du parc informatique et des utilisateurs de Stormshield}
                {
                \begin{itemize}
					\item \textit{Gestion de l'infrastructure de Stormshield :}
					\tabto{0.3cm} Réflexion et mise en place de soltions permettant d'améliorer l'infrastructure des sites de Stormshield, Gestion des projets de mise en place de nouveaux services informatiques, Mise en place et gestion des sauvegardes, Gestion de la maintenance des logiciels et matèriels, Gestion d'incidents utilisateurs et des postes de travails, Gestion des partenaires et des achats.\\
					\textcolor{gray}{\emph{\scriptsize{Compétence : Réseaux, pare-feu, Virtualisation, système}}}\\
					\textcolor{gray}{\emph{\scriptsize{Technologie : Stormshield, linux, Windows, powershell, bash, VmWare, glpi, apache, mysql, Microsoft Exchange Server, Zimbra, Microsoft Active Directory, GPO}}}
                 \end{itemize}
		} 	
	\cventry{2007--2008}{Prestataire de Service Informatique}{FlowLine}{Clermont-Ferrand}{}{}{}{}
\section{Langues}
\cvitemwithcomment{Anglais}{Lu, parlé, écrit}{}

\section{Compétences informatiques}
\cvdoubleitem{Réseaux}{blabla, blabla}{Système}{blabla, blabla}
\cvdoubleitem{Virtualisation}{blabla, blabla}{gestion de sources}{blabla, blabla}
\cvdoubleitem{\LaTeX}{blabla, blabla}{Python}{blabla, blabla}

\section{Centres d'intérêts}
	\cvitem{}{SnowBoard, Roller, Planche à voile, Kite, Jonglage divers, Rubik's Cube}
\end{document}
