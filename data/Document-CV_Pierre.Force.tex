\documentclass[10pt,a4paper,sans]{moderncv}
\moderncvstyle{casual} 
\moderncvcolor{blue}
\setlength{\hintscolumnwidth}{2cm}
\usepackage[utf8]{inputenc}
\usepackage[scale=0.9, top=0.3cm, bottom=2.5cm]{geometry}
\usepackage{helvet}
\usepackage[french]{babel}
\usepackage [T1] {fontenc}
\frenchsetup{ItemLabels=\textendash}
\frenchsetup{ItemLabeli=\textbullet}
\usepackage{xspace}
\usepackage{tabto}
\newcommand{\g}[1]{\og #1 \fg}
\definecolor{myblue}{rgb}{0.22,0.45,0.70}

\name{Pierre}{FORCE}
\title{Ingénieur Qualité} 
\address{57, place des mimosas}{69\,290 Craponne}{France}
\phone[mobile]{06~09~05~69~71}
\email{force.pierre@gmail.com}
\photo[64pt][0.4pt]{data/pierreforce.jpg}

\begin{document}
\makecvtitle
\section{Formation}
	\cventry{2003--2006}{Diplôme Universitaire de technologie}{IUT université Clermont Auvergne}{Clermont-Ferrand (63)}{}{Spécialité Réseaux et Télécommunication}
	\cventry{2000--2003}{baccalauréat Série Scientifique}{Lycée Godefroy de Bouillon}{Clermont-Ferrand (63)}{}{Option Science de l'ingénieur}
\section{Experience professionnelle}
	\cventry{2016--2019}{Ingénieur Qualité}{Stormshield}{Lyon (69)}{Développement d'un logiciel d'administration centralisé de pare-feu : \g{Stormshield Management Center} (SMC)}
		{
			\begin{itemize}
			\item Validation fonctionnelle du produit SMC :\\
			\tabto{0.5cm} \textcolor{gray}{\emph{(prouver le bon fonctionnement du produit dans évolution et diffuser son fonctionement)}}
				\begin{itemize}
				\item Réflexion et mise en place de maquettes de réseaux complexes
				\item Écriture de tests automatiques (scripting)
				\item écriture de plans et scénario de tests manuels (rédaction)
				\end{itemize}
			\end{itemize}
			\begin{itemize}
			\item Réflexion et mise en place de méthode de travail dans le cadre de la méthode \g{Agile Scrum}:
				\begin{itemize}
				\item Gestion du quotidien et intégration: \g{Scrum Master}
				\item Réflexion et mise en place de nouvelles procédures afin d'améliorer et d'optimiser la qualité du produit.
				\end{itemize}
			\end{itemize}
		}
	\cventry{2008--2016}{Administrateur Systèmes et réseaux}{Stormshield}{Lyon (69)}{Administration, maintenance et supervision du parc informatique et des utilisateurs de Stormshield}
                {
                    \begin{itemize}
                    \item Gestion d'incidents utilisateurs (helpdesk) :
                             \begin{itemize}
                              \item Priorisation, communication et intervention sur les postes de travail de l'ensemble des employés Stormshield (R\&D et commerciaux)
                             \end{itemize}
                     \end{itemize}
                     \begin{itemize}
                     \item Gestion de l'infrastructure de Stormshield :
                             \begin{itemize}
                             \item Réflexion et mise en place de solutions permettant d'améliorer l'infrastructure complète de Stormshield pour l'ensemble des services.
                             \item mise en place et gestion des sauvegarde de l'ensemble des données informatiques de Stormshield.
                             \item gestion de la maintenance logiciel et matériel de l'infrastructure de Stormshield (mises à jour de sécurité, renouvellement des garanties matériels)
                             \item Gestion des achats de matériel et logiciel de Stormshield
                             \end{itemize}
                     \end{itemize}
		} 	
	\cventry{2007--2008}{Prestataire de Service Informatique}{FlowLine}{Clermont-Ferrand}{}{}{}{}
\section{Langues}
\cvitemwithcomment{Anglais}{Lu, parlé, écrit}{}
\section{Compétences informatiques}
\cvdoubleitem{Réseaux}{blabla, blabla}{Système}{blabla, blabla}
\cvdoubleitem{Virtualisation}{blabla, blabla}{gestion de sources}{blabla, blabla}
\cvdoubleitem{\LaTeX}{blabla, blabla}{Python}{blabla, blabla}
\section{Centres d'intérêts}
\cvitem{Sports de glisse}{SnowBoard, Roller, Planche à voile, Kit}
\cvitem{Jonglage}{Diabolo, bolas, bâton, bâton du diable, balles, massues}
\cvitem{Rubik's Cube}{3x3x3 : 32 secondes}
\end{document}
